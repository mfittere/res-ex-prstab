\documentclass[aps
,prstab
%,reprint
%,draft
,preprint,tightenlines
%,linenumbers
%,longbibliography
%,preprintnumbers
%,showpacs
%,showkeys
,amsfonts,amssymb,amsmath
%,floatfix
]{revtex4-1}

\usepackage{bm}
\usepackage{dcolumn}
\usepackage{graphicx}
\usepackage[dvipsnames]{xcolor}
\usepackage[colorlinks=true, allcolors=NavyBlue]{hyperref}

\usepackage{mathptmx}
\linespread{1.1}            
\usepackage[T1]{fontenc}

\newcommand{\q}[2]{\ensuremath{#1\ \mathrm{#2}}} % quantity with units
\newcommand{\code}[1]{\textsc{#1}} % computer code
\newcommand{\comm}[1]{\emph{#1}} % editorial or referee comment
\newcommand{\note}[1]{\emph{\textcolor{red}{#1}}} % note, to-do

% abbreviations
\newcommand{\kth}{$k$th}
\newcommand{\kthtp}{$k$th-turn pulsing}
\newcommand{\seventhtp}{7th-turn pulsing}
\newcommand{\eighthtp}{8th-turn pulsing}
\newcommand{\tenthtp}{10th-turn pulsing}

\newcommand{\Fma}{Frequency-map analysis}
\newcommand{\fma}{frequency-map analysis}

\begin{document}

\title{Responses to the comments of editors and referees}

\date{\today}

\preprint{Phys. Rev. Accel. Beams, manuscript ZR10120 Fitterer}

\maketitle

Dear Editors,

thank you for the editorial comments and referee reports.

The paper was thoroughly revised, taking those suggestions into account.

Responses to all recommendations and criticisms are reported below.

\section*{Editorial comments}

\comm{
We also suggest that Reference [5] should be updated to a journal article:

[5] M. Benedikt, F. Zimmermann, ``Towards future circular colliders,''
Journal of the Korean Physical Society (2016)

and perhaps in addition to include the recent IPAC paper
  
[5b] M. Benedikt and F. Zimmermann,
``FCC: Colliders at the Energy Frontier,'' 
Proc. IPAC2018, Vancouver.
}

\note{(Giulio)}

\comm{In reviewing the figures of your paper, we note that the following
changes would be needed in order for your figures to conform to the
style of the Physical Review.  Please check all figures for the
following problems and make appropriate changes in the text of the
paper itself wherever needed for consistency.

Figure(s) [3,13,21,25,29]
              The abbreviation "a.u." is used to indicate atomic units only;
          arbitrary units are abbreviated "arb.  units"; "A.U." is
          for astronomical units; "amu" is for atomic mass units.
          Please amend your figures and manuscript text accordingly.
  
Figure(s) [4]
               The size and font of the letters and numbers appearing in
          figures should be uniform (or close to it).  Please make the
          letters and numbers more consistent.
  
Figure(s) [4,11]
              Please change E1, E2, etc., to power of 10 notation (i.e.,
          10 with appropriate superscript).  We do not print computer
          notation.
  
Figure(s) [9,10,16,23,26]
              Please modify quantities so that superscripts and subscripts are
          set as they would appear in text. Subscripts should be printed
          below the line. Also, we do not print computer notation. For
          example, set powers as superscript numbers; do not use ^2,
          **2, etc., for squares of quantities. Please adjust the text
          of the paper accordingly.
  
Figure(s) [9,10,16,23,26]
               The lettering in the axis labels and/or numbering size should
          be increased.
  
Figure(s) [13,21,25,29]
               Please rearrange power of 10 in axis label for clarity:  Either
          (i) place the power of 10 as a factor, without parentheses, in
          front of the axis label quantity, changing the sign of the power
          as needed; or (ii) incorporate the power of 10 in the topmost
          or rightmost number on the scale.  Please refer to the URL
          http://journals.aps.org/authors/axis-labels-and-scales-on-graphs-h18
          for a pictorial representation of the preferred forms for
          axis labels.
  
Figure(s) [23,26]
              Please insert post-decimal zeros (1.0, not 1.) or delete the
          decimal point.
        }

\note{(Miriam)}


TITLE:

This journal does print title footnotes.  Move informatio into your
acknowledgments section.

The following problems were noted in your manuscript.

* Please check the changes made to the title of your manuscript.
  We propose a modified and somewhat shorter title.

----------------------------------------------------------------------
Report of Referee A -- ZR10120/Fitterer
----------------------------------------------------------------------

REFEREE A - ZR10120 

The paper describes quite extensive simulation and experimental work 
in estimating the effects of either resonant or random dipole 
excitations on the LHC proton beam at injection. Quantities studied 
include beam loss, transverse beam profiles, and bunch length. The 
studies are motivated by the planned use of a hollow electron lens in 
the HL-LHC in pulsed mode (either resonantly with the betatron tune, 
or randomly). The pulsed mode should deplete the beam halo but not 
affect the beam core. 

The paper is well written, with an excellent introduction, high 
quality tables and figure (a few issues are mentioned below), and very 
few typos. It reports in substantial detail on work performed in 2016 
and 2017. 

My main criticism is with the discussion and conclusion. In this 
section, the previous sections are summarized as one would expect, but 
the authors may be able to come to more pronounced conclusions. The 
following sentence is somewhat disappointing after such a large body 
of work: "However, further work is needed to identify, both in 
simulations and experimentally, candidate excitation patterns that are 
effective for halo removal and that preserve the beam core ...". 

Would it be possible, for example, to summarize the beam losses of all 
cases studied experimentally and in simulation in a single table? What 
conclusions could be drawn from such a direct comparison for the cases 
studied? Are any of these excitation pattern viable for a hollow 
electron lens in the HL-LHC - which was the motivation for this work 
after all? 

Typos and minor comments: 

- Table I: "... for past, present and future colliders." Here the 
authors only show the colliders at their labs. The Tevatron is really 
only an example of 3 colliders with comparable stores energy, and the 
ISR (not mentioned at all) had up to several MJ of stored energy too. 

- Table III: "$\mu perv$" is not an SI unit. Use 
"$\mu$\text{AV}^{-2/3}$". 

- Table IV: MOF and MOD are not defined 

- line 472: "can thus attributed" => "can thus be attributed" 

- all figures: Since the paper reports on both measurements and 
simulations it would help the reader if the captions would start with 
either "Measurement ..." or "Simulation of ..." or "Calculation of 
..." 

- Figure 8: Can "random" be a dashed line for better distinction? 
Also, the emittance growth for "random" should be calculable (e.g. 
Handbook 1999, Syphers, Sec. 4.5.6 "Emittance growth"). 

- Fig. 30, 31: it is difficult to distinguish the symbols for "damper 
on" and "damper off" 

- Ref. [15] - A better or additional reference may be X. Gu, PRAB 20, 
023501, 2017. 

- Ref. [41] is not used in the text - there may be more unused 
references 

- there may be more references on emittance growth - although I could 
not find a specific one on resonant excitation, as the authors have 
pointed out

----------------------------------------------------------------------
Report of Referee B -- ZR10120/Fitterer
----------------------------------------------------------------------

REFEREE B - ZR10120 

In this article various types of beam excitations are numerically and 
experimentally investigated, which act on proton losses, beam 
emittances, and beam distributions in the LHC. These studies are 
important for the design of hollow electron lenses (HEL) for active 
beam halo control of HL-LHC, where a considerable amount of energy 
will be stored in the beam tails. 

Given the need of active halo control for HL-LHC and for future 
high-power accelerators such as HE-LHC and FCC-hh these investigations 
are very essential. The experimental work and simulation results 
presented in this article are excellent. In my view, the article 
contains too many technical details, especially too many excitation 
schemes that lead to confusing plots (see Figs. 17, 18, 22, 27, 28 and 
32), with many detailed results that may not be of great interest for 
the general audience. This detailed information will certainly be very 
useful for colleagues working on high-power hadron colliders and 
should be complied in a technical note or elsewhere. Also in some of 
the FMA diagrams (see Figs. 9, 10 and 16) the difference between the 
various excitation modes are barely visible. 

Sometimes plot sizes are just too small and in other cases, the 
difference between various settings seem to be marginal and does not 
contain any further findings. 

Since this work is very relevant for the design and upgrade of future 
high-power hadron accelerators and the performed work is excellent, I 
suggest reducing the length and content of the paper to a more compact 
and comprehensible level. In particular, the number of excitation 
schemes displayed in individual plots should be reduced to a level 
that the content of the various plots is clearly recognizable.

\end{document}
